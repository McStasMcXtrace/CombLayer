\subsubsection{ESS}
\label{sec:model:ess}

\paragraph{Beam lines}
\label{sec:model:ess:beam-lines}

\subparagraph{StopPoint variable}
\label{sec:model:ess:beam-lines:StopPoint}
The {\tt StopPoint} variable stops building a beam line at a given point:
\begin{description}
\item[0] builds the whole beam line
\item[1] builds nothing more than a bit up to the \SI{5.5}{\meter} mark
\item[2] builds everything in the bunker but not through the bunker wall
\item[3] builds up to the end of the Bunker
\item[4] builds up to the cave %hatch
\item[5] bulds everything beyond the cave (object where the sample/detector are located)
\end{description}

If you use the {\tt essBeamLine} executable then you can also give the start point
and then you will only get a fragment of your instrument.
Obviously, you would get nothing if the start and end point are the same.
\alert{syntax of ./essBeamline?}