This section describes how to use CombLayer from a user's (i.e. non-developer) point of view.
It is focused on the ESS model, which can be generated by running

\begin{bash}
  ./ess -r a
\end{bash}

This command produces the \mcnp input file {\tt a1.x} as well as two other files: {\tt ObjectRegister.txt} and {\tt Renumber.txt}.

\subsection{Variables}
In the beginning of the input file there is a commented list of variables which define the geometry:

\begin{deck}
 c ----------------------------------------------
 c --------------- VARIABLE CARDS ---------------
 c ----------------------------------------------
 c ABunkerFloorDepth 120
 c ABunkerFloorThick 100
 c ABunkerLeftAngle 0
 c ABunkerLeftPhase -65
 c ABunkerNLayers 1
 ...
\end{deck}

The variable name consists of the component name and its corresponding parameter. For instance,
the first variable {\tt ABunkerFloorDepth} in the list above sets the floor depth of the component called {\tt ABunker}.

\subsubsection{How to change variables}
Any of these variables can be changed either via a command line arguments or an XML file.

Let's change the Beryllium reflector height.
First of all, we need to find out which variable we need to change and therefore find out the name of the Be reflector
component in CombLayer.

To do this, open the \mcnp geometry and click on any Be reflector cell. In my case it's cell number 5~(exact number depends on the
CombLayer version you are using).

Now we need to find out which component this cell belongs to.  Find this cell number in the {\tt Renumber.txt} file:
\begin{bash}
grep " 5 " Renumber.txt 
Surf Change:1000006 5                                                           
Cell Changed :1000005 5 Object:BeRef (topBe)       
\end{bash}
It shows that the corresponding Be reflector object is called {\tt BeRef}.

Now we need to find out which {\tt BeRef} variable is responsible for its height:
\begin{bash}
grep BeRef a1.x 
c BeRefHeight 74.2
c BeRefLowRefMat Be5H2O
c BeRefLowWallMat Stainless304
c BeRefRadius 34.3
c BeRefTargSepMat Void
c BeRefTopRefMat Be5H2O
c BeRefTopWallMat Stainless304
c BeRefWallThick 3
c BeRefWallThickLow 0
\end{bash}

We can guess from this list that the variable we need is called {\tt BeRefHeight}.

\paragraph[Command line]{Changing variables via command line}
In order to change a variable via command line arguments, run:
\begin{bash}
  ./ess -r -v BeRefHeight 50 a
\end{bash}
Several variables can be changed, e.g.:
\begin{bash}
  ./ess -r -v BeRefHeight 50 -v BeRefRadius 35 a
\end{bash}

\paragraph[XML file]{Changing variables with XML file}
Create an XML file with the following content:

\begin{xml}
<?xml version="1.0" encoding="ISO-8859-1" ?>
<metadata_entry>
  <Variables>
    <variable name="BeRefHeight" type="double">50</variable>
  </Variables>
</metadata_entry>
\end{xml}

and generate the modified geometry:
\begin{bash}
  ./ess -r -x model.xml a
\end{bash}

All the variables can be exported in the XML file by running
\begin{bash}
  ./ess -r -X a
\end{bash}
