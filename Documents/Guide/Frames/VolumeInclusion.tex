%\section{HowTo}
\begin{itemize}
\item Get real surface number by its relative number: SMap.realSurf(divIndex+103) (see createLinks methods)
\end{itemize}

\subsection{How to put one object into another}

Suppose, we are inserting Spoon into Mug.
Mug is made up of N cells. Spoon is made of one contained component with outer surface.
CombLayer provides several methods to put one object into another:

\begin{cpp}{Name=essBuild}{floatplacement=H}
attachSystem::addToInsertForced(System,   *Mug, *Spoon);
attachSystem::addToInsertSurfCtrl(System, *Mug, *Spoon);
attachSystem::addToInsertControl(System,  *Mug, *Spoon);
attachSystem::addToInsertLineCtrl(System, *Mug, *Spoon);
\end{cpp}

\subsubsection{addToInsertForced}

The outer surface of the Spoon is excluded from the HeadRule of every single cell of Mug.
Even if Mug contains cells which do not intersect with Spoon (e.g. its handle, see \figref{fig:forced}).
{\it Forced} means {\it do it and do not think about it}, but at the same time it means that
{\it I have got something wrong somewhere.}  Normally this is that insufficient link points have been added
to the object, or that the object is a set of split (single cell) volumes.  However, there is the additional
problem that the model may not be correctly constructed at this point, so that the other options seem not to work.
This can be checked by adding a
\prog{SimProcess::writeIndexSim(System,"OutputFilename.txt",0);}
in the code just before the call to insertForced. If there are undefined volumes then the model is not in a state that
any of the \prog{addToInsert} algorithms except \prog{addToInsertForced} can be used. 

\begin{figure}
  \centering
  \begin{tikzpicture}
    \drawMug{0}{0}{}
    \drawSpoon{2}{1}{dashed}
  \end{tikzpicture}
  \caption{addToInsertForced: all cells of Spoon are excluded from all cells of Mug including non intersecting cells}
  \label{fig:forced}
\end{figure}


\subsubsection{addToInsertSurfCtrl}

The objective of this function is to use the surface intersections between Mug and spoon to determine which 
cells within Mug intersect the ContainedComp of spoon. The process is done on a cell - ContainedComp level. 

The process is as follows:
\begin{enumerate}

\item{Deconvolves both the Spoon's containedComponent boundary into surfaces.}
\item{Loop over each cell in Mug : MCell}
  \begin{enumerate}
  \item{Calculate intersection of each surface:surface:surface triplet from the set of CC surfaces and MCell surface}
  \item{If a point is within CC and the MCell exclude the CC from the
    MCell and goto next cell}
  \end{enumerate}
\end{enumerate}
   
Thus the spoon is inserted only into those cells of Mug which it intersects (\figref{fig:surfctrl}).

It is not always better to call {\tt addToInsertSurfCtrl} instead of
{\tt addToInsertForced} in cases that if is certain that an
intersection can must take place (particularly if the CC / Inserting
cells have large numbers of surfaces).

{\tt addToInsertSurfCtrl} is a very expensive function to call,
because you have to check all the surface triplets. So, it runs a bit
slower than addToInsertForced, but the geometry will be faster.

\begin{figure}
  \centering
  \subfloat[Cell 3 is excluded from cell 1; cells 2 and 4 not changed]{
    \begin{tikzpicture}
      \drawMug{0}{0}{}
      \drawSpoon{2}{4}{dashed}
      \node[legend] at (2,2) {1};
      \node[legend] at (-1.25,2) {2};

      \node[legend] at (2.25,4.5) {3};
      \node[legend] at (2.25,6) {4};
    \end{tikzpicture}
  }
  \subfloat[None of the cells are modified]{
    \begin{tikzpicture}
      \drawMug{0}{0}{}
      \drawSpoon{5}{1}{dashed}
      \node[legend] at (2,2) {1};
      \node[legend] at (-1.25,2) {2};

      \node[legend] at (5.25,2) {3};
      \node[legend] at (5.25,3) {4};
    \end{tikzpicture}
  }
  \caption{addToInsertSurfCtrl: only needed cells are excluded}
  \label{fig:surfctrl}
\end{figure}

The two remaining methods provide similar functionality but with less
computational overhead, however, there are cell constructs which will
cause them to fail.


\subsubsection{addToInsertControl}

It's a very simple method. The link points from Spoon are uses as a test for each of the cells within Mug. 
The method checks if any of these link point fit inside each of the cells of Mug. If it does, then it cuts Spoon from the Mug cell.
It is possible to add a vector of link points to check as a parameter to limit the search.

\subsubsection{addToInsertLineCtrl}

Imagine we have a (big) contained component~(Mug) and some (small)
object which clips it~(Spoon). The link points are {\bf not} in the
Mug~(therefore {\bf addToInsertControl} can not be used), but the
lines which connect them are in the Mug.  The method checks the lines
connecting the link points and sorts out the intersections.
