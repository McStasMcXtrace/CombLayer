%\section{HowTo}
\begin{itemize}
\item Get real surface number by its relative number: SMap.realSurf(divIndex+103) (see createLinks methods)
\end{itemize}

\subsection{How to put one object into another}

Suppose, we are inserting Spoon into Mug.
Mug is made up of N cells. Spoon is made of one contained component with outer surface.
CombLayer provides several methods to put one object into another:

\begin{cpp}{Name=essBuild}{floatplacement=H}
attachSystem::addToInsertForced(System,   *Mug, *Spoon);
attachSystem::addToInsertSurfCtrl(System, *Mug, *Spoon);
attachSystem::addToInsertControl(System,  *Mug, *Spoon);
attachSystem::addToInsertLineCtrl(System, *Mug, *Spoon);
\end{cpp}

\subsubsection{addToInsertForced}

The outer surface of the Spoon is excluded from the HeadRule of every single cell of Mug.
Even if Mug contains cells which do not intersect with Spoon (e.g. its handle, see \figref{fig:forced}).
{\it Forced} means {\it do it and do not think about it}, but at the same time it means that
{\it I have got something wrong somewhere.}  Normally this is that insufficient link points have been added
to the object, or that the object is a set of split (single cell) volumes.  However, there is the additional
problem that the model may not be correctly constructed at this point, so that the other options seem not to work.
This can be checked by adding a
\prog{SimProcess::writeIndexSim(System,"OutputFilename.txt",0);}
in the code just before the call to insertForced. If there are undefined volumes then the model is not in a state that
any of the \prog{addToInsert} algorithms except \prog{addToInsertForced} can be used. 

\begin{figure}
  \centering
  \begin{tikzpicture}
    \drawMug{0}{0}{}
    \drawSpoon{2}{1}{dashed}
  \end{tikzpicture}
  \caption{addToInsertForced: all cells of Spoon are excluded from all cells of Mug including non intersecting cells. In this example, the Spoon is excluded from the Mug handle even though
  they do not intersect.}
  \label{fig:forced}
\end{figure}


\subsubsection{addToInsertSurfCtrl}

The objective of this function is to use the surface intersections between Mug and spoon to determine which 
cells within Mug intersect the ContainedComp of spoon. The process is done on a cell - ContainedComp level. 

The process is as follows:
\begin{enumerate}

\item{Deconvolves both the Spoon's containedComponent boundary into surfaces.}
\item Deconvolves the Mug into surfaces.
\item{Triple Loops over surfaces of both Mug and Spoon}
  \begin{enumerate}
  \item{Calculate intersection of each surface:surface:surface triplet}
  \item{If a point is within CC and the Mug Cells exclude the CC from the
    MCell}
  \end{enumerate}
\end{enumerate}
   
Thus the spoon is inserted only into those cells of Mug which it intersects~(\figref{fig:surfctrl}).

It is not always better to call {\tt addToInsertSurfCtrl} instead of
{\tt addToInsertForced} in cases that if is certain that an
intersection can must take place (particularly if the CC / Inserting
cells have large numbers of surfaces).

{\tt addToInsertSurfCtrl} is a very expensive function to call,
because you have to check all the surface triplets. So, it runs significantly
slower than addToInsertForced, but the geometry will be faster.

\begin{figure}
  \centering
  \subfloat[Cell 3 is excluded from cell 1; cells 2 and 4 not changed]{
    \begin{tikzpicture}
      \drawMug{0}{0}{}
      \drawSpoon{2}{4}{dashed}
      \node[legend] at (2,2) {1};
      \node[legend] at (-1.25,2) {2};

      \node[legend,blue,fill=white] at (2.25,4.5) {3};
      \node[legend,blue,fill=white] at (2.25,6) {4};
    \end{tikzpicture}
  } \quad
  \subfloat[None of the cells are modified]{
    \begin{tikzpicture}
      \drawMug{0}{0}{}
      \drawSpoon{5}{1}{dashed}
      \node[legend] at (2,2) {1};
      \node[legend] at (-1.25,2) {2};

      \node[legend,blue,fill=white] at (5.25,2) {3};
      \node[legend,blue,fill=white] at (5.25,3) {4};
    \end{tikzpicture}
  }
  \caption{addToInsertSurfCtrl: only needed cells are excluded}
  \label{fig:surfctrl}
\end{figure}

The two remaining methods provide similar functionality but with less
computational overhead, however, there are cell constructs which will
cause them to fail.


\subsubsection{addToInsertControl}

It's a very simple method. The link points from Spoon are uses as a test for each of the cells within Mug. 
The method checks if any of these link point fit inside each of the cells of Mug. If it does, then it cuts Spoon from the Mug cell~(\figref{fig:control}).
It is possible to add a vector of link points to check as a parameter to limit the search.

\begin{figure}
  \centering
  \subfloat[Cell 3 is excluded from cell 1 since cell 3 has a link point inside cell 1.
  Cell 4 is excluded from cell 1 since Mug has a link point which is located inside cell 4.
  Cells 2 is not modifed.]{
    \begin{tikzpicture}
      \drawMug{0}{0}{}
      \node[legend] at (2,2) {1};
      \node[legend] at (-1.25,2) {2};
      \draw[red] (2,5) circle (2pt);

      \drawSpoon{1.8}{3}{dashed}
      \node[legend,blue,fill=white] at (2.05,3.5) {3};
      \node[legend,blue,fill=white] at (2.05,5.5) {4};
      \draw[blue] (2.05,3) circle (2pt);
    \end{tikzpicture}
  } \quad
  \subfloat[Intersection is not made because none of the link points are located in another component.
  Geometry is broken. In order to fix it,
  either define more link points or use another volume inclusion method.]{
    \begin{tikzpicture}
      \drawMug{0}{0}{}
      \node[legend] at (2,2) {1};
      \node[legend] at (-1.25,2) {2};
      \draw[red] (-1.53,2) circle (2pt);
      \draw[red] (-1,2) circle (2pt);
      \draw[red] (0,2.5) circle (2pt);

      \drawSpoon{-0.85}{-1}{dashed}
      \node[legend,blue,fill=white] at (-0.6,-0.3) {3};
      \node[legend,blue,fill=white] at (-0.6,3.5) {4};
      \draw[blue] (-0.6,4) circle (2pt);
      \draw[blue] (-0.6,0.5) circle (2pt);
    \end{tikzpicture}
  }
  \caption{addToInsertControl: intersection relies on the link points}
  \label{fig:control}
\end{figure}

\subsubsection{addToInsertLineCtrl}

If  we have a (big) contained component~(Mug) and some (small)
object which clips it~(Spoon) and we are not certain that the link points 
are contained within each Mug cell, then  {\bf addToInsertControl} can not be reliably used.
However, addToInsertLineCtrl adds an additional check to addToInsertControl, because it 
constructs all the connecting lines between the link points, and if the line 
intersects the other cell, an intersection is made~(\figref{fig:linectrl}).

\begin{figure}
  \centering
  \subfloat[Intersection is not made because the line connecting two link points intersects
  cell 4 outside of a cell belonging to the Mug]{
    \begin{tikzpicture}
      \drawMug{0}{0}{}
      \node[legend] at (2,2) {1};
      \node[legend] at (-1.25,2) {2};
      \draw[red] (-1.53,2) circle (2pt);
      \draw[red] (-1,2) circle (2pt);
      \draw[red] (0,2.5) circle (2pt);

      \drawSpoon{-0.85}{-1}{dashed}
      \node[legend,blue,fill=white] at (-0.6,-0.3) {3};
      \node[legend,blue,fill=white] at (-0.7,3.5) {4};

     \draw[olive] (-1.53,2) -- (0,2.5);
    \end{tikzpicture}
  } \quad
  \subfloat[Cell 4 is excluded from cell 2 since at least one line connecting two
link points intersects the Mug handle at a point which is valid within the spoon 
(shown by red filled circles)]{
    \begin{tikzpicture}
      \drawMug{0}{0}{}
      \node[legend] at (2,2) {1};
      \node[legend] at (-1.25,2) {2};
      \draw[red] (-1.53,2) circle (2pt);
      \draw[red] (-1,2) circle (2pt);
      \draw[red] (0,2.5) circle (2pt);

      \drawSpoon{-0.85}{-1}{dashed}
      \node[legend,blue,fill=white] at (-0.6,-0.3) {3};
      \node[legend,blue,fill=white] at (-0.7,3.5) {4};
      \draw[blue] (-0.6,4) circle (2pt);
      \draw[blue] (-0.6,0.5) circle (2pt);

     \draw[olive] (-0.6,0.5) -- (-0.6,4);
     \draw[olive] (-1.53,2) -- (0,2.5); 

     \draw[olive] (-1.53,2) -- (-0.6,4);

     \draw[fill=red] (-0.6,3.27) circle (3pt);
     \draw[fill=red] (-0.6,2.8) circle (3pt);
     \draw[fill=red] (-0.6,1.21) circle (3pt);
     \draw[fill=red] (-0.6,0.8) circle (3pt);

   \end{tikzpicture}
  }
  \caption{addToInsertLineCtrl: intersection relies on the lines connecting link points (not all link points and connections are shown).}
  \label{fig:linectrl}
\end{figure}
